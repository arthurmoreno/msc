%{{{ Preamble
\documentclass[12pt,a4paper,letterpaper]{article}
%}}}
%{{{ Packages
%{{{ geometry
\usepackage[
    bottom=2cm,
    left=3cm,
    right=2cm,
    top=3cm,
]{geometry}
%}}}
%{{{ bibliography
\usepackage[
    backend=bibtex,
    defernumbers=true,
    encoding=latin1,
    style=abnt,
]{biblatex}
\usepackage[autostyle]{csquotes}
%}}}
%{{{ type input font
\usepackage[T1]{fontenc}
\usepackage[brazil]{babel}
\usepackage[brazil]{varioref}
\usepackage[utf8]{inputenc}
%}}}
%{{{ type output font
\usepackage{
    amsfonts,
    amsmath,
    amsopn,
    amssymb,
    amsthm,
    latexsym
}
\usepackage{indentfirst}
%}}}
%}}}
%{{{ Add bib
\addbibresource{bib/articles.bib}
%}}}
%{{{ Add New Command
\newcommand\wb[1]{\discretionary{#1}{#1}{#1}}
%}}}
%{{{ Documento
\begin{document}
\section{Pesquisa por artigos de algoritmo de classificação}

Foram feitas buscas de trabalhos relacionados a palavra\wb-chave
``\textit{algoritmos de classificações}'', nas seguintes fontes Google Scholar e
Periódico Capes, foram escolhidos artigos de dois anos mais recentes, logo de
forma grosseira usamos o critério de escolha devidos aos seus títulos, onde
foram selecionados e lidos resumos, no entanto selecionamos devido a nossa busca
dois artigos relacionados com algoritmo de classificação.

\section{Artigos Escolhidos pelo título}

Primeiro artigo, foi buscado por palavra\wb-chave ``\textit{algorithm for
classification dataset}'', filtrado por ano 2017 e encontramos um artigo com
título ``\textit{Feature weighting and SVM parameters optimization based on
genetic algorithms for classification problems}'' \autocite{PHAN2017}.

Nosso segundo artigo, foi buscado por palavra\wb-chave ``\textit{algorithm for
classification image}'', filtrado por ano 2016 e encontramos um artigo com
título ``\textit{Fast image classification by boosting fuzzy classifiers}''
\autocite{KORYTKOWSKI2016}.

\section{Primeiro Artigo}
Neste artigo foi utilizado algoritmo genético e máquina de vetor de suporte de
no para otimização na classificação de problemas, em \autocite{PHAN2017}.
Foram utilizados 11 problemas de classificação oriundo da base de dados publicas
da UCI. Comparada com métodos de classificações do estado da arte, onde se saiu
bem competitiva com os demais.

\section{Segundo Artigo}
\autocite{KORYTKOWSKI2016}

\pagebreak
%{{{ Referências Bibliográficas
\medskip
\printbibliography[
    heading=bibintoc,
    title={Referências Bibliográficas}
]
%}}}
\end{document}
%}}}
